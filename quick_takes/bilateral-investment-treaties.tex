\addchap{Trading Sovereignty For Credibility: The Power Of MNCs}
\label{ch:bilateral-investment-treaties}

\initial{T}he first letter is special



   The first bilateral investment treaty (BIT) was signed in 1959 between
   Germany and Pakistan. 60 years later, more than 3,000 BITs form a
   governance network for international investment. BITs were initially
   intended to assure foreign investors that they will benefit from the
   same standard of governance in a host state as they would in their home
   state, regardless of the quality of a host state's legal systems,
   particularly in cases of expropriation.

   BITs usually include binding dispute settlement clauses, which allow
   foreign investors to bypass domestic legal routes in favour of
   international arbitration. Developing countries are motivated to sign
   BITs because they are assumed to make the state a more attractive
   designation for foreign direct investment, and that this will have
   beneficial spillovers with regards to development, economic growth and
   good governance. However, arbitration procedures allow foreign
   investors to undermine host states good governance efforts by
   subverting domestic institutions.

   Moreover, since the early 2000's, the regime has become increasingly
   contentious in the wake of high-profile international investment
   arbitration cases, which have increased states' awareness of the
   potential and unintended ramifications of signing BITs. During this
   time there was an exponential increase in the number of disputes
   brought forward by investors against a host state, where the average
   award sought has been upwards of 500 million USD, although claims have
   been filed for several billion USD.

   The nature of the disputes brought forward has been particularly
   controversial, with BITs used to induce regulatory chill, or cause an
   altogether reversal of public policy. BITs can and have been used by
   investors to challenge, in arbitration, legitimate government policies
   concerning public health, the environment, safety, security, cultural
   diversity, and financial services. Indonesia and Costa Rica have been
   identified as having abandoned environmental measures due to the threat
   of potential investment arbitration; it is impossible to know how many
   states have abandoned policy even before an arbitration claim has been
   filed. Professor Eric Neumayer has argued that in signing BITs,
   developing states have essentially traded `sovereignty for
   credibility.'

   However, it is not only developing countries who are feeling the
   ramifications of the BITs they have signed. Two of the most publicized
   disputes were actually filed against Australia and Germany. In Philip
   Morris v Australia, the tobacco giant tried to force Australia to
   overturn plain-packaging laws. They were ultimately unsuccessful -- the
   laws were deemed a legitimate public health measure -- but the dispute
   did cause several other countries to await the result before
   instituting similar laws. In Vattenfall v Germany, the Swedish firm
   filed a claim for several billion USD after nuclear production was shut
   down in Germany following the Fukushima Daiichi nuclear disaster in
   2011. These cases have forced developed countries to also reconsider
   their approach to governing foreign investors.

   There are few examples of states that have successfully withdrawn from
   BITs; South Africa is a powerful one. Fearing arbitration claims in the
   wake of the establishment of its Black Economic Empowerment initiative,
   which mandated partial black ownership of all enterprises, the state
   introduced the Protection of Investment Act (2015), which provides a
   system of non-discrimination by formally granting the same protections
   to both foreign and domestic investors.

   Due to the provision and threat of international arbitration, BITs
   consistently undermine the ability of developing states to craft policy
   and regulate foreign investors. The fundamental necessity of investment
   treaties, at least for developing states where there are strong
   institutions (such as domestic courts) that can adjudicate disputes,
   should be called into question.

   [21]Rachael HerzMarch 20, 2019[22]bilateral investment treaty,
