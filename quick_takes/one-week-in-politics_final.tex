\addchap{Is One Week A Long Time In Politics?}
\label{ch:one-week-in-politics}

{\flushright\largegreycapitals{Freddie Kellett}}
\smallskip

\initial{T}his year marks the 30th anniversary of the Berlin wall falling. Other
   members of the Roosevelt Club and I have been reflecting on this
   momentous event, and on what has happened since then. In searching for
   parallels between the geopolitical tensions of the `a week is a long
   time in politics.'

   To anyone following Brexit this adage would seem as applicable as ever.
   But if we look under the surface, at the content of political
   ideologies, this impression begins to unravel. Everyone is familiar
   with Donald Trump's famous wall---the one that doesn't even exist yet,
   and may never exist. This proposal has been a corner stone of his
   election campaign, but it really isn't a new concept. Would Trump have
   learnt about Hadrian's wall at school?

   More seriously though, it is easy to be engrossed in the short-term,
   fast-paced nature of politics, forgetting that there is a longer
   context within which these issues fit. This is partly the fault of
   journalism, an enterprise that wishes to distil issues to their bare
   minimum, but it is equally attributable to our own need for news to be
   open-and-shut, with clear causes and consequences.

   Anyone who has watched Yes Minister will have been (perhaps not so-)
   shocked by the similarities to political issues and personal dramas of
   the present day. The series' politicians may have lived 50 years ago,
   but we are both talking about trident, budget cuts, national security,
   and strife in the nation's political leadership.

   It may be tempting to brush off the current political trends for
   populism and the rejection of expert opinion as new faces for old
   problems. I have been pondering this for some time, but I can't say it
   is as simple as that. Populism pitches people against elites; conflict
   between the two is deep, and in some cases well-founded. The process of
   globalisation and trade liberalisation initiated in the 1980s has been
   great for big business, but it has left behind some (rather large)
   segments of society. Even a glance at some basic economic indicators
   will tell you that these problems have been brewing for a while. Who
   knows how long populism will last? What we do know is that it has
   already had longer than a week.

   It is not surprising that such a simple adage is not applicable to
   current political reality. But this mantra does still have value. It
   serves to show is that even though we are in uncharted territory, there
   is a rhythm to politics. There may be significant political events
   occurring, which can make even a day seem like a long time. For the
   most part these are interspersed with smaller, repeated, and benign
   problems that politics has been dealing with for centuries. Workers
   rights, civil unrest, budgets, these will always be the bread and
   butter of politics.

   Political commentators try to make our present situation a soap opera.
   Rolling news stories, and click-bait headlines---it is all too easy to
   be drawn into this thinking. It is much harder to see events in a
   longer view. Wilson may have been right, but perspective and distance
   seem hard to come by right now. This is something we should bear in
   mind.

