\addchap{Renewable Energy: All Talk?}
\label{ch:renewable-energy-tech}

\initial{T}he first letter is special



With so many ways to cut back on carbon pollution, why aren't we?

   "12 years left..." I have been hearing that a lot lately -- in
   political discussions, Sustainable Development classes, and on social
   media. It is a hot topic, particularly energised by America's left-wing
   politicians (Alexandria Ocasio-Cortez comes to mind). To many, this is
   a political issue; I don't think so.

   Climate scientists say we have 12 years left before we hit the
   1.5�C rise in temperature that the Paris Accord warned us
   about. Politicians use this as ammo for pushing their green
   agenda. While there is scientific merit to this prediction, there is
   always a healthy amount of skepticism one should retain. Over a decade
   ago, former Vice-President Al Gore was touting that the world only had
   10 more years until we went over the tipping point -- yet here we are
   with another dozen years to go. While there may be disagreement over
   this doomsday timeline, however, it is clearly coming.

   Even if the Earth does not become completely uninhabitable in
   12 years, we are all still affected. Even those of us in favourable
   areas (away from coastlines and more arid zones) are already suffering
   inconveniences: avid scuba divers will be heartbroken to know that over
   half of the Great Barrier Reef has died since 2016; skiers are faced
   with unpredictably warm winters and deadly storms -- all of which will
   continue to grow in intensity.

   Luckily -- or so one would think -- renewable energy has made great
   strides in recent years. Unfortunately, many politicians appear to see
   climate change more as a way to garner attention than a deadly trend
   they want to combat. Many of the improvements in renewable energy
   technologies should be more widespread.

   Small communities have been able, in some cases, to go carbon-neutral.
   Dubai -- possibly the world's most unnatural city -- went from desert
   to a conglomerate of fake islands, harbours, and skyscrapers in just a
   few decades. In 2006 it boasted of the most carbon pollution per
   capita. In spite of this, it includes a carbon-neutral town,
   Sustainable City, which spans 113 acres, and has 500 villas, nearly 100
   apartment buildings, and over 1 million feet of office and retail
   space. If a zero-carbon town can exist in an area as inhospitable as
   the desert, it can ostensibly be reproduced even even more easily in
   other places.

   Renewable energy is not just beneficial to our environment: it offers
   great energy solutions for developing countries. Several companies, for
   instance, make products to utilise water for energy. Enomad is a cheap
   portable hydrogenator that can fit in a backpack and be placed in
   streams, or even dragged behind a kayak in order to build
   up a charge to power a phone. Even large-scale, prefabricated
   hydrogenators are available and only require a small stream with a
   slight change in elevation. Turbulent offers just such devices; they
   can provide up to 15 kW of power, and can be assembled within a few
   days of unskilled labor. One of their projects powers
   a 400-student school in Bali, and a local farm. Even if one is
   skeptical of climate change, the power of renewable energy can be
   highly valuable to rural or developing communities.

   Whether or not the planet cuts down on emissions, a lot of damage has
   been done. Again, there are companies making strides in this
   department. Swiss company Climeworks creates filters that suck carbon
   out of the air. Their product claims to be 400 times more efficient
   than afforestation, uses no water, and is relatively
   affordable. Catalytic Innovations turn carbon dioxide and water into
   usable biofuel to replace other harmful products. Several groups are
   making headway towards reducing pollution.

   While society continues to pump harmful greenhouse gasses into the
   atmosphere, we seem to be more obsessed with talking about the problem
   than solving it. Entrepreneurs and academia have done more to mitigate
   this than governments. People have shrieked at the impending doom for
   years, yet there are clear improvements that industry is not pouring
   enough resources into. Politicians need to stop using "Green" as an
   election sound-bite, and not only make real investments in the field,
   but use the technology already at our disposal.

FURTHER Reading:

   Gore, Will. "'We Have 12 Years to Act on Climate Change before the
   World as We Know It Is Lost. How Much More Urgent Can It Get?'" The
   Independent. October 08, 2018. Accessed March 17,
   2019. [21]https://www.independent.co.uk/voices/climate-change-ipcc-envi
   ronment-paris-agreement-global-warming-a8573811.html.

   Freedland, Jonathan. "Interview: Al Gore on His Mission to save the
   Planet." The Guardian. May 31, 2006. Accessed March 17,
   2019. [22]https://www.theguardian.com/film/2006/may/31/usa.environment.

   Kunzig, Robert. "The World's Most Improbable Green City." National
   Geographic. April 4, 2017. Accessed March 17,
   2019. [23]https://www.nationalgeographic.com/environment/urban-expediti
   ons/green-buildings/dubai-ecological-footprint-sustainable-urban-city/.

   Harris, Mark. "The Entrepreneurs Turning Carbon Dioxide into Fuels."
   The Guardian. September 14, 2017. Accessed March 17, 2019.
   https://www.theguardian.com/sustainable-business/2017/sep/14/entreprene
   urs-turn-carbon-dioxide-into-fuels-artificial-photosynthesis.

   [24]Energy[25]Dain RohtlaApril 3, 2019[26]renewable energy, [27]green
