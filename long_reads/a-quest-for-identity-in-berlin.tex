\addchap{A Quest for Identity in Berlin}
\label{ch:a-quest-for-identity-in-berlin}

\initial{T}he first letter is special



   "Ich bin ein Berliner," "I am a Berliner" declared President John. F.
   Kennedy on June 26, 1963, a time when Berlin citizens themselves had
   trouble defining what it meant to be a "Berliner." The question of
   identity has been at the core of Berlin's history and is still of
   utmost relevance today, 30 years after the fall of the Wall.  Indeed,
   it allows to explain why people who seemed so close, were yet so far.
   Despite being from the same country, the same city, the divide imposed
   radically different lifestyles to people and the influence of Regime's
   ideologies on urban planning, architecture and culture indirectly
   shaped identities, family stories, and heritages in different ways.

   The fall of the Berlin Wall, in November 1989, allowed the obliteration
   of the most obvious traces of separation between the Western and the
   Eastern parts of the city. Roads were reconnected, public transports
   linking the two halves of Berlin were brought back into operation, and
   abandoned places such as the Potsdamer Platz, which became a no-man's
   land after World War II, were retrofitted and rehabilitated . However,
   despite such undeniable material reunification, in the years following
   1990 (date on which the reunification of East and West Germany was made
   official), a virtual wall remained. It is only starting to dissipate
   today, with the arrival of new generations who did not witnessed life
   in a divided city and whose references are based on parent's and
   grandparent's stories as well as harmonized history courses.

   In order to understand those less evident but yet significant
   disparities, one must look at the scale of the adaptations that were
   made because of the wall construction in terms of urban planning. In
   1961, the wall was still only a succession of barriers, guards, houses,
   and coils of barbed wire, but public transportation between the East
   and the West was already turned away if not severed. Migrations from
   one side to the other gradually became impossible. West Berliners who
   already experienced a blockade in June 1948, were left surrounded by a
   red sea of communism and developed a quasi "island mentality" within
   their capitalist bubble. Berlin became the ideological battlefield of
   the Cold War and among the most powerful and efficient arms used were
   architectural and structural choices. Indeed, the conflict is
   well-known for its space-race, or its nuclear arm-race, but the case of
   the Berlin Wall  also shows that an "architecture race" was also in
   order. From 1961 to 1989, both the German Democratic Republic (GDR) and
   the Federal Republic of Germany (FRG) undertook massive construction
   plans:

   On the West side, US-Marshall Plan aids sustained projects such as the
   creation of shopping centers, the reconstruction of entire districts
   (Hansa) and of buildings such as the Congress Hall and the American
   Memorial Library. Capitalist values of cosmopolitanism, openness and
   abundance were promoted with the help of renowned architects like Peter
   Eisenman or Rem Koolhaas, through the use of materials like steel and
   glass, and more generally through the incorporation of a very modernist
   style in the city's busy areas. Streets were also named after key
   figures of the Allies forces (Avenue Charles de Gaulle).

   Such physical transformations also aimed at entrenching Western culture
   in Berlin. Policies allowing to lift the curfew in bars and clubs were
   for instance sat up, and the Berlin Opera was rebuilt in order to give
   an image of a "never-sleeping," young, West Berlin.

   On the East side things were quite different. As the given prime
   purpose of the wall was to keep western "fascists" away,
   ideologically-motivated architecture also arose. Following socialist
   policies, social housing complex were built and homes were
   standardized. A plan to turn East Berlin into an impressive capital
   city was adopted thus triggering the construction of imposing buildings
   like the TV tower, the Kino International or the Hotel Berolina all in
   "socialist realism" styles. The Soviets also organized a tram network
   which purpose was to advocate shared transportation. It was conceived
   to counter the massive investment into cars that West-Berlin encouraged
   as cars were a symbol of freedom.

   All those examples of physical developments of each side constitute a
   perfect example of Foucauldian patterns of power construction as they
   were in both cases utilized to assert political thoughts and, to a
   certain extent, shape population's identities. It therefore becomes
   clear why the fall of the wall as such was not enough for Berliners to
   easily unify and find a sense of common identity.

   The notion of wall transcends the simple definition of "border made out
   of bricks or concrete to prevent movement of people," as it conveys
   deeper ideological, political and even philosophical meanings. Life in
   what used to be East and West Berlin was besides still experienced
   differently even after the wall's demolition. The question of abortion
   for instance is an interesting one: Abortion was legal in the GDR and
   illegal in the FRG. As the two Germanies unified, an agreement was
   signed stating that the situation was to remain unchanged temporarily
   despite the quasi-disappearance of physical borders.

   A nation's common identity is often constructed from collective
   memories or shared history. Some generations of Berliners did
   experience both the construction and the fall of the Wall however, this
   does not necessarily mean that their Cold War memories are collective
   ones. In fact, one can even go as far as to argue that old assumptions
   over each side of the city, that were successfully transmitted to
   citizens through means of soft power, still endure.

   Berlin has gone through a lot of changes since the fall of the Wall:
   the night-life scene shifted to the East following the rise of new
   music styles, streets were renamed, houses were renovated, the arrival
   of the Euro currency harmonized Berlin's economic system, and abandoned
   spaces left room for innovation. Despite it all, attempts to build a
   "New Berlin" will not erase the past. The Wall's impacts go far beyond
   practical ones and even affected citizens psychologically. The
   importance of remembering its history only arose recently as many
   people still considered it to be a symbol of soviet repression. It is
   also curious to note that there are more pieces of the Wall currently
   on display in the United States then there are in Berlin.

   There are different dimensions to the construction and fall of the
   Wall. From geographical, to political, or cultural, such dimensions all
   allow to better understand Berlin's history and the reasons why the
   city holds today, such exclusive strategical position in terms of
   European and International affairs. Berlin is a polycentric city which
   gradually manages to successfully combine diverging memories and to
   unify into one "New" city. It could therefore constitute a legitimate
   small-scale model for an arguably fragmented Europe in which
   inequalities between East and West remain.

