\addchap{"Open the gate!"}
\label{ch:open-the-gate}
%The fall of the Berlin Wall, Thirty Years On
\initial{I}n the summer of 1989, Francis Fukuyama predicted "the end of
   history"--the overall triumph of the principles of liberal democracy,
   advanced by the West, against the trappings of communism. While the end
   of all conflict was not an idea that anyone, even Fukuyama, believed
   in, the sheer force of the economic and ideological factors which
   overturned one half of the world's most dangerous duo--the Soviet
   Union--was inspirational to the core. Supporting this tide of optimism
   were the events in the autumn of the very same year. At the heart of
   Europe, the Berlin Wall was pulled apart by the hands of East and West
   Germans alike, who chanted, "Tor auf!" or "Open the gate!" The coming
   months would bring an overwhelmingly peaceful revocation of
   Marxism-Leninism, and the opening of the Soviet bloc's `gates' to the
   rest of the world.

   Exactly thirty years later, the dream of that November night in 1989 is
   slowly crumbling. The severance of local and regional ties--not to
   mention the abandonment of international agreements--has been a
   hallmark of the past few years. Each region of the world has a
   veritable selection-box of crises. In Europe, the Brexit predicament is
   underscored by the Syrian refugee crisis, and a resulting lack of
   compliance with the European Union's policies across its member states.
   More localised problems span from separatist movements in Scotland and
   Catalonia to a new threat of division in an Ireland which has only
   recently achieved a tenuous peace. In the Americas, while Venezuelans
   starve under the political contest between two leaders, Donald Trump
   continually presses forward on the building of a border wall between
   the United States and Mexico. He rallies his supporters with identity
   politics of inclusion and exclusion, constructing an exaggerated `them'
   threatening a country whose economy, in reality, relies heavily on
   immigrant labour to continue functioning. Tensions between India and
   Pakistan have just recently escalated to claim the lives of dozens of
   citizens in a conflict over the territory of Kashmir. The religious
   undertones of this conflict are echoed in other areas of the world,
   where Muslim extremists practice jihad and chant "Death to America."

   Of course, the world of 1989 was not perfect. Amid the optimism, there
   was always suffering--and there has been much progress since,
   especially for those of us in the West, who now enjoy a standard of
   living unrivalled by that of the late 1980s. But, in the words of the
   Guardian's John Harris: whatever happened to the future? In the early
   years after the fall of the Berlin Wall, Europe saw incredible
   movements promising peace, unification, and a new, bold idealism. The
   Good Friday agreement provided a solution for decades of violent
   conflict in Ireland, while Europe as a whole managed to consolidate
   itself. It incorporated Eastern European countries into the European
   Union and established multilateral treaties such as the Schengen
   agreement, which all but abolished borders between select countries.
   Full economic integration looked like a real possibility with the
   creation of the Eurozone and the resulting common currency, the Euro.

   Outside of Europe, apartheid was abolished in South Africa and the
   Israelis and Palestinians attempted dialogue--both seemed unfathomable
   in years prior. The common belief was in the victory of liberalism and
   in hope for the future, spurred on by revolutions in technology and
   communications. The era was not without its crises and catastrophes--it
   would be wholly unjust to ignore the Rwandan genocide, the Balkan
   conflicts, the American-led involvement in the Middle East, and
   countless other sources of misery in order to paint an idyllic picture
   of the post-Cold War era. On the whole, however, Western hope seemed
   consummate. In recent years, this hope has begun to break down. Rafael
   Behr, another Guardian columnist, writes that "The west that won the
   cold war no longer exists." What happened to it?

   Behr points to an artificial feeling of instability and insecurity
   among the richest countries on Earth--those of Western Europe and the
   United States. In fairness, this feeling has been fed by a series of
   crises--the attacks of September 11, 2001, the global financial
   collapse of 2008, and the recent terrorist violence in Paris and
   Manchester, to name a few. Our environment is predicted to radically
   change due to human-created climate change in the coming decades--a
   fact which some choose to label a hoax. Economically there is a
   widespread feeling of a market disproportionately dominated by China,
   and politically, the interference of Russian agents in the world's
   loudest and proudest democracy is shocking to the core. These issues
   are just a start on the long list of Western grievances.

   On an individual level, globalisation and modernisation have
   contributed to an erosion of traditional feelings of community. Many
   have lost faith in their political institutions, which allow the
   ultra-rich to become richer while basic needs like a living wage and
   proper, accessible healthcare services are neglected. In this
   situation, we retreat to an "orgy of reminiscence," remembering our
   heavily-romanticised national pasts, clinging to them, and fashioning
   communities we feel can protect us from the real pain we have suffered
   over the past decades. We have defied logic, polls and predictions,
   causing shocking upsets in, for example, the 2016 presidential
   election, and the Brexit vote called by David Cameron in the same
   year.

   A profound pessimism about the future thus prevails in the West. While
   the issues that our countries face are very real, the insecurities that
   have taken them over are a product of a privileged past that became
   idealised and universalised. The West managed to fashion an age of
   idealism out of an event that was, at its heart, European. The fall of
   the Berlin Wall had global implications, yes, but non-Western countries
   all over the globe experienced the same hardships and threats both
   before and after the events of 1989, forcing them to remain realistic
   and grounded. Even in the present day, problems that originated in or
   even before the twentieth century continue to resonate in Korea, China,
   South Africa, Afghanistan, and countless others.

   The West is coming out of its "end of history" into a world which
   recognised no such high period. This `new normal' is being addressed by
   fashioning barriers, and pitting some against others. Looking to
   history, this strategy has not worked or lasted--the isolationism of
   the interwar period, for example, was enough to drag the world into a
   second world war, after which world leaders realised the merits and
   necessity of cooperation and integration. Above all, in the modern age,
   this integration is not just a goal to strive for--it is a fact of
   life. The revolutions in communication and transportation technology
   that have taken place in the last few decades have allowed for a new
   level of economic integration across the world, and even a significant
   level of social and cultural integration. Borders and boundaries cannot
   stop the tide of modernity. The West must overcome its insecurity, set
   aside its prejudice, and enforce cooperation. This may not lead to an
   age of idealism like that which many believe characterised the years
   after 1989--in fact, it almost certainly will not do so. Nevertheless,
   the decision-makers of today must have the courage to open gates in a
   world that cannot tolerate walls.

Further Reading

   [21]https://www.theguardian.com/commentisfree/2019/jan/01/1989-eastern-
   europe-britain-optimism
   [22]https://www.theguardian.com/commentisfree/2018/jan/10/west-cold-war
   -capitalism-eastern-bloc-populism

   [23]https://www.theglobalist.com/the-breach-of-the-berlin-wall/

   Fukuyama, Francis. "The End of History?" The National Interest, no. 16
   (1989): 3-18. http://www.jstor.org/stable/24027184.

