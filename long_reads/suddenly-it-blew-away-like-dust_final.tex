\addchap{Suddenly It Blew Away Like Dust}
\label{ch:suddenly-it-blew-away-like-dust}

\initial{T}hirty years ago, the Berlin Wall came down. Nobody saw it coming. `I
   remember in 1989,' said the famed Indian novelist Salman Rushdie,
   `...had you said it was going to happen no one would have believed you.
   The system seemed powerful and unbreakable. Suddenly overnight it blew
   away like dust.' Yet their total failure to foreshadow such a momentous
   event did not stop Western observers from making a shocking number of
   bold statements which purported to describe the future course of world
   history.

   The American historian Francis Fukuyama famously proclaimed the `End of
   History' in 1989, arguing that with the fall of the USSR's `Evil
   Empire' and the collapse of Communism as a viable model of government,
   there was only one alternative: the entire world would adopt liberal
   democracy along American lines. Every country would elect a parliament,
   enjoy free enterprise and free trade, enforce equality under the law,
   and so forth. Fukuyama remained influential far longer than he deserved
   to be. He worked for the hugely prestigious RAND corporation and as a
   policy planner for Clinton?s State Department. It wasn't until the
   World Trade Centres fell in 2001 that people caught on. History was
   very much alive and kicking.

   Of course, had Western leaders been paying attention they would have
   realised this blindingly obvious fact a decade earlier. When the Soviet
   Union and other Communist super-states crumbled in the early nineties,
   Eastern Europe degenerated into a cycle of gruesome violence and
   brutality not seen since the Second World War. Albanians, Croatians,
   Bosnians Muslims, Serbs -- a whole host of ethnic groups whose regional
   identities were subsumed under the Communist Yugoslav state declared
   themselves independent and committed the most heinous atrocities
   against each other in the process. One Serbian army force alone killed
   or ethnically cleansed tens of thousands of Bosnian Muslims in what is
   regarded as the first European genocide since the Holocaust. It was
   only contained, to some extent, by the bombing of the guilty party,
   conducted under the aegis of the North Atlantic Treaty Organisation
   (NATO) and spearheaded by none other than President Clinton himself.

   Indeed, even in Russia proper the inevitable march towards liberal
   democracy went awry almost immediately. The new nation's first
   democratic election had to be heavily `influenced' by the U.S.
   government and the International Monetary Fund to avoid a Communist
   Party victory, the teams of highly-educated experts sent by the U.S. to
   oversee Russia's transition to capitalism obviously bungled it: the
   country quickly fell into the hands of mobsters and oligarchs, a
   situation which remains relatively unchanged today. Capitalism did, in
   the end, come to Russia; but it was a crony capitalism of an
   unfriendly, un-democratic sort.

   China, too, watched the fall of the Berlin Wall with great interest.
   They did not so much like what they saw. When pro-democracy activists
   took to the streets, they emulated Khrushchev, not Gorbachev; they sent
   in the tanks. (This event, known as the Tiananmen Square Massacre,
   occurred only months after Fukuyama proclaimed the end of history and
   killed an estimated 10,000 people.) To be sure, this isn't to say that
   China hasn't changed with the times. In fact, they are providing a
   model the world over for the fusion of cutting-edge technology and
   brutal totalitarianism. The government's use of super-accurate facial
   recognition technology, the introduction of an almost comically tragic
   Black Mirror-esque `Social Credit' system, and the massive
   concentration-camp-like `political re-education centres' intended to
   Sinocise the country's millions of Muslim citizens are achievements of
   which Hitler and Stalin would never have dared dream.

   If the leaders of the nineties saw the future as a bland utopia of
   globalised liberalism, what do the leaders of today predict? Generally,
   doom and gloom. It doesn't matter whether you're on the left or right:
   the future is not looking good. If you voted for Trump, for Brexit, for
   Le Pen or Orban, you probably think Western civilisation is buckling
   under the pressure of economic decline and automation, unchecked
   immigration and cultural change, and the threat of Islamic terrorism.
   If you voted for Clinton or Corbyn, Sanders or Melenchon, you probably
   think that runaway wealth inequality, climate change, and the rise of
   right-wing populism are going to wreck decades of incremental steps
   towards a better future. The point is that nobody seems to be
   optimistic about the future, which, after examining the evidence, seems
   rather reasonable.

   But I hope I have made you wary of predictions of any kind, no matter
   how well-founded they may seem. To try and foreshadow the unknowable
   future is folly. We can extrapolate from current trends, to be sure:
   for example, `If we continue to burn fossil fuels at the present rate,
   the planet will be two degrees warmer by 2100.' But there is only one
   prediction we can reliably make, now and forever: history will never
   end. Even the tallest and most unbreakable walls can, in a single,
   shocking, unbelievable moment, blow away like dust.

