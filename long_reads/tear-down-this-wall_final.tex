\addchap{"Tear Down this Wall!"}
\label{ch:tear-down-this-wall}
%\textbf{An Anatomy of the Barriers that Divide Us}

% Define new format just for this article that starts with F
\renewcommand{\initial}[1]{%
     \lettrine[lines=3,lraise=0.18,lhang=0.45,findent=5pt,nindent=0em]{
     				\color{rooseveltblue}
     				{\textsf{#1}}}{}}

{\flushright\largegreycapitals{Gustavo Viegas}}
\smallskip

\initial{F}ew expected the events of November 9th, 1989. Pressure had been
   mounting for months and civil unrest had spread across the
   coal-polluted cities of East Germany, but when the regime's spokesman,
   G{\"u}nter Schabowski, mistakenly announced that East Germans would be free
   to cross the border, it was over. Western media subsequently proclaimed
   open borders, crowds gathered at checkpoints, and one of the most
   iconic symbols of human history fell, not by the powder of artillery or
   the radioactive waves of a nuclear Armageddon, but through the peaceful
   and sudden change in people's minds. Walls are not just pieces of
   concrete, steel and armed guards in their watchtowers. They symbolise
   something greater and can show curious observers like yourselves
   something about the social, political and economic factors shaping a
   region and its people. As the 30th anniversary of the fall of the
   Berlin Wall rapidly approaches, the Cold War is no more. Soviet
   totalitarianism is no more. The Iron Curtain is no more. But walls
   still play a role in the human psyche, both metaphorically and
   physically. We shall look at three examples of walls in three different
   places, understand their meaning, question their effectiveness, then
   engage in a thought experiment about a world in which hard borders are
   non-existent and humans are allowed to move freely between states.

   It may be an obvious fact that walls are man-made,
   artificially-introduced objects, but the implications of such a simple
   statement are considerably more complex. It means that there is no
   inherent value or meaning in walls, only those introduced by human
   perception. It also means that there is nothing natural about walls, or
   indeed about borders and states. As Benedict Anderson famously
   advocated, these are `imagined communities.' There is no way you can
   taste, touch, or feel the United States of America. True, contrary to
   nations and states, walls are physical objects, captured and
   interpreted by our senses. But their whole purpose is to mark, protect
   or even expand territory and border lines drawn and imagined by humans
   through their long history of conflict, war and diplomacy. Most of the
   Berlin Wall was still physically intact after the events of November
   9th, and yet, everyone refers to that day as the day of its fall. In
   other words, the Cold War symbol was defined much more in terms of its
   subjective, humanly attributed meaning than the objective agglomeration
   of concrete and barbed wire that composed it. Indeed, when this meaning
   ceased to exist, so did the physical and geographical existence of the
   wall, which was demolished shortly afterwards, as the two Germanies
   reunited.

   This leads us to our first example: The Great Wall of China. It has not
   been active for millennia, given that Mongolian invaders no longer pose
   a threat. Somehow, it still has meaning, but this meaning is not
   necessarily the same as it was 2000 years ago, thus highlighting a
   second important point about walls: their meaning can change over time
   -- it is not fixed in stone, as the walls themselves are. To illustrate
   this, let us get back to our example. The Great Wall of China is
   actually a series of walls spanning around 6000 kilometres. Some of
   them date back to the sixth century B.C., but the best-preserved ones
   were built during the Ming dynasty (1368-1644). Their main objective
   was to keep the northern nomadic herders out, as they craved the
   south's wealth and luxury. After Genghis Khan unified the Mongols and
   conquered China, his son established a new dynasty, the Yuan, but
   subsequent peasant revolts led to its fall and the rise of the Ming
   dynasty, who then decided to put an end to northern incursions. They
   were convinced that the massive fortification would inhibit nomadic
   aggression as well as resist the erosive effects of weather and time --
   and they spent vast amounts of resources to put their plan into action.
   Northern incursions did continue, however, and the low morale and harsh
   living conditions for troops stationed at the wall led some of the
   Chinese defenders to socialise and even collude with northern
   combatants. Thus, the military effectiveness of the wall was
   questionable, if that. The structure finally shed its old meaning when
   the Ming's successors, the Manchu (or Qing), expanded Chinese territory
   northwards, making the wall unnecessary. But then something happened:
   the wall assumed new meaning. It became a symbol of China's ancient
   power and history, as well as engineering and architecture. It was
   classified as a UNESCO World Heritage site and one of the Seven Wonders
   of the World. It remains the world's largest military structure and an
   invocation of human ingenuity. It also attracts around 10 million
   visitors every year, boosting revenue from tourism. In other words, the
   wall is no longer used to keep people out; it is used to bring them in.
   Military relevance gave way to symbolism, archaeological history, and
   economic development as the new defining elements of one of the world's
   most ancient structures.

   Rather than defending against bloodthirsty armies of invaders, many
   modern walls are built simply to curb unauthorised migration. As you
   read this, you are probably thinking of the headline-grabbing US-Mexico
   wall, and the attention it recently got from Trump's political
   ascension. However, the case I would like to discuss is the less-known,
   but equally important border fence built between the Spanish
   North-African territory of Melilla and Morocco. It is one of the most
   fortified borders on the planet. The barrier is divided into a series
   of layers. First, a tall metal fence, followed by a tilted fence; then
   barbed wire, another tall fence with more barbed wire, and a flexible
   top section. On the Moroccan side, you will find a ditch followed by
   more fences. Everything is monitored by surveillance cameras and
   guards. Most of the immigrants come from Sub-Saharan Africa. They flee
   war, persecution and economic hardship. One of their strategies is to
   gather a big group and focalise their advance into one punctual point
   of the fence, overwhelming the guards and trying to reach Spanish
   territory where they are guaranteed certain protections by European
   law. Some are sent to migrant centres where they are safe from
   immediate deportation, others are immediately driven back and badly
   hurt by the guards. After the refugee crisis of 2014-5, both Spain and
   Morocco increased their defences at the border, reducing the number of
   trespassers significantly. The wall thus symbolises a division between
   rich and poor, former colonies and former metropoles, and its meaning
   is stronger than ever, as a wave of right-wing populism and xenophobia
   infects European politics. What is interesting is that no matter how
   big the fence is, people will continue to come. Their desperation is so
   great that facing a wall head-on, hiding under vehicles or walking into
   oversaturated boats become logical, even necessary steps.

   For an article commemorating the fall of the Berlin Wall, this last
   example seems fitting: it is also a symbol of the Cold War, and a
   persistent one. The wall -- a border barrier in the form a
   demilitarised buffer zone -- between the northern Democratic Republic
   of Korea and the southern Republic of Korea seems like a relic from a
   past age. A product of US-Soviet disputes in East Asia after years of
   Japanese colonialism, a World War, and the bloody Korean War of 1950-3,
   the border drawn at the 38th parallel breaks one nation into two
   ideologically opposed states. As the Soviet Union disintegrated and
   China turned to state capitalism, the North Korean regime refused to
   compromise, doubling down on its pursuit of nuclear weapons and its
   open hostility towards the US and its allies. In the meantime, the
   differences between each side of the border barrier were only
   accentuated as South Korea's Asian-Tiger-style economic miracle became
   as real as a Samsung smartphone and a Hyundai SUV -- while famine and
   state control over the economy became the norm in the North. Just like
   the families that were permanently separated because their members
   happened to be on different sides when the wall was erected, the Korean
   peninsula has learnt to live with a division that is not just between
   rich and poor, technological and primitive, democratic and
   totalitarian; more than anything, it is between open and closed. It is to this openness that we now turn.

   For all the talk of the history and meaning of walls, this article's
   mission has been to break them. Its title is taken from a certain
   American President who, in a famous speech to Berliners in 1982,
   exclaimed: "Mr. Gorbachev, tear down this wall!" If walls are dependent
   upon the subjective meanings people give them, a change in
   people's minds could confine them to the history books. In that
   spirit, \textit{The Economist} published an article in
   2017 imagining a world where borders were open. And by open, it did not
   mean to spell out the end of the nation-state, it simply meant open: goods,
   services, capital and especially people would be able to move freely
   across states. Workers would go to where they are most productive,
   getting payed larger salaries by richer clients, reducing labour waste
   and making the world probably trillions of dollars richer. Latin
   Americans would flee from urban violence, and Sub-Saharan Africans from
   murderous dictators, making the idea of open borders a moral one. Since
   the North's institutions are hard to replicate in the Global South, a
   family moving from Bolivia to the Netherlands seems like a much more
   straightforward way of fighting absolute poverty. This is not to say
   that emigrating is easy, but that the option should be there for those
   who wish to (or must) pursue it.

   We do not live in the age of the Ming dynasty, and Mongol invaders have
   nothing in common with Syrian refugees. The reason why the largest
   economy in the world became large in the first place is because it went
   from a few million white settlers and black slaves in agrarian 1800, to
   320 million diverse and mostly urban citizens in high-tech 2019. The
   development of a country's economic and political institutions is also
   hurt by the number of walls erected to `protect' the same institutions.
   Progress is more than just building bridges. It is about tearing down
   walls.

% Back to default from roosevelt.cls
\renewcommand{\initial}[1]{%
     \lettrine[lines=3,lraise=0.18,lhang=0.33,findent=5pt,nindent=0em]{
     				\color{rooseveltblue}
     				{\textsf{#1}}}{}}
