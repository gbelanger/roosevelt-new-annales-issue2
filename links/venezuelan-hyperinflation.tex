   #[1]RSS Feed

   [2]The Roosevelt Group
   (BUTTON)

   [3]Home [4]Name [5]Mission (BUTTON) Publications [6]People

   [7]Join
   (BUTTON) Back [8]Our Journal [9]Quick-Takes [10]Book Reviews
   (BUTTON)

   ____________________ (BUTTON)

   [11]Home[12]Name[13]Mission [14]Publications [15]Our Journal
   [16]Quick-Takes [17]Book Reviews [18]People
   [19]The Roosevelt Group

   [20]Join

Fighting Hyperinflation in Venezuela

Someone call the firefighters!


   Imagine you are a German citizen in October 1923. Every time you go to
   a pub after work, you buy two pitchers of beer at once. The second gets
   warm even before you finish drinking the first, and no one likes warm
   beer, but you do it anyway. Why? Because the loss in the value of that
   second beer as it gets warm is nothing compared to the loss in the
   value of the money laying idle in your wallet. If you had waited just a
   few hours, you would not have been able to afford the second beer at
   all: prices are skyrocketing so fast that your whole purchasing power
   lies in the dust, and your economic well-being along with it. Welcome
   to the world of hyperinflation.

   Although the world record still belongs to post-war Hungary, where
   hyperinflation reached nearly 42 quadrillion per cent, Venezuela's
   hyperinflation is predicted to reach 10 million per cent this year,
   according to the IMF. This is bad enough. The primary cause of this
   catastrophic economic affliction is excessive growth in the supply of
   money. When governments have inadequate tax systems and lack the
   credibility to raise funds by issuing debt, some central banks,
   influenced by malicious political figures, might make the fatal
   decision to energise their printing presses. Prices begin to rise
   abruptly. People expect inflation to get worse, so they buy even more
   in the short run to avoid higher prices in the future, which in turn
   worsens hyperinflation, as excess demand takes prices sky-high. Worse
   still: since there is a delay between a tax being levied and being
   payed, even if it is just a few months, by the time the government gets
   the money, its value has plummeted, thereby greatly reducing tax
   revenue and vastly increasing the government deficit. A fiscal and
   monetary fiasco thus ensues in the form of a vicious cycle. Add to the
   equation an autocratic dictatorship with a record of flagrant human
   rights violations and you have Venezuela.

   Under most circumstances, you would expect the government to take the
   necessary steps to curb hyperinflation once prices start to rise and
   the currency starts to depreciate. But when this government happens to
   be a fragile dictatorial regime highly dependent on the political
   interests of a few groups in society, socioeconomic reform might not be
   its primary interest, so the situation is allowed to continue. Until,
   of course, mobs start to gather outside the gates of the palace. At
   this point, however, the little credibility the government had before
   it oversaw the nation's descent into the hyperinflationary scourge is
   gone.

   The first step to resolving the crisis, therefore, is regime change.
   People need to believe that the government is serious about addressing
   the problem, since people's expectations of inflation are an essential
   part of minimising the damage. Then, the new government needs to launch
   a major fiscal reform pack that will limit government spending (which
   in Venezuela accounts for an unsustainable 30% of GDP) and raise taxes.
   This would reduce the deficit and stabilise the fiscal situation. Now a
   credible reshuffle of personnel needs to be pursued in the Treasury and
   central bank so that the public believes that the printing presses will
   be put to rest, and an anti-inflationary, contractionary monetary
   policy will be enacted. Moreover, it is vital to establish some sort of
   anchor to the currency (such as dollarization) so that people have some
   sign that the transition is credible. Finally, and especially in the
   dismal case of Venezuela, outside assistance is paramount: the
   government must negotiate a recovery programme with the IMF so that
   international investors will regain their confidence, and the
   government will be able to access international credit markets as an
   additional source of finance, other than taxes.

   The anti-hyperinflationary recipe is as clear as the economic science
   behind it. But in Bolivarian Venezuela it is the ideologues that rule,
   not the technocrats. This implies that the final ingredients must be
   found in the political sciences: Maduro has to go.

   [21]Gustavo ViegasMarch 13, 2019[22]venezuela, [23]hyperinflation,
   [24]maduro, [25]economics, [26]politics

   Previous

Trading Sovereignty For Credibility: The Power Of MNCs

   Rachael HerzMarch 20, 2019bilateral investment treaty, economics,
   international law, multinational corporations
   Next

Is One Week A Long Time In Politics?

   Freddie KellettMarch 6, 2019politics, brexit

   [27]Contact Us
   [28]Members area

   [29]Home[30]Name[31]Mission[32]Publications[33]People

References

   Visible links:
   1. https://www.roosevelt-group.org/quick-takes?format=rss
   2. https://www.roosevelt-group.org/
   3. https://www.roosevelt-group.org/home
   4. https://www.roosevelt-group.org/name
   5. https://www.roosevelt-group.org/home/#mission
   6. https://www.roosevelt-group.org/people
   7. https://www.roosevelt-group.org/join
   8. https://www.roosevelt-group.org/publications-2/#new-annales-index
   9. https://www.roosevelt-group.org/publications-2/#quick-takes-index
  10. https://www.roosevelt-group.org/publications-2/#reviews-index
  11. https://www.roosevelt-group.org/home
  12. https://www.roosevelt-group.org/name
  13. https://www.roosevelt-group.org/home/#mission
  14. https://www.roosevelt-group.org/publications-1
  15. https://www.roosevelt-group.org/publications-2/#new-annales-index
  16. https://www.roosevelt-group.org/publications-2/#quick-takes-index
  17. https://www.roosevelt-group.org/publications-2/#reviews-index
  18. https://www.roosevelt-group.org/people
  19. https://www.roosevelt-group.org/
  20. https://www.roosevelt-group.org/join
  21. https://www.roosevelt-group.org/quick-takes?author=5c8862ee41920263418dfe55
  22. https://www.roosevelt-group.org/quick-takes/tag/venezuela
  23. https://www.roosevelt-group.org/quick-takes/tag/hyperinflation
  24. https://www.roosevelt-group.org/quick-takes/tag/maduro
  25. https://www.roosevelt-group.org/quick-takes/tag/economics
  26. https://www.roosevelt-group.org/quick-takes/tag/politics
  27. https://www.roosevelt-group.org/contact-us
  28. https://www.roosevelt-group.org/members
  29. https://www.roosevelt-group.org/home
  30. https://www.roosevelt-group.org/name
  31. https://www.roosevelt-group.org/mission
  32. https://www.roosevelt-group.org/new-annales-2
  33. https://www.roosevelt-group.org/people

   Hidden links:
  35. https://www.roosevelt-group.org/search
  36. http://instagram.com/therooseveltclub
  37. http://www.facebook.com/354310218289537
  38. https://www.roosevelt-group.org/quick-takes/bilateral-investment-treaties
  39. https://www.roosevelt-group.org/quick-takes/one-week-in-politics
