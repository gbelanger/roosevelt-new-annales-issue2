\addchap{Principles}
\label{ch:principles}

\initial{I}would not be surprised if you hadn't heard of Ray Dalio -- he is not
   some celebrity businessman like Musk, Zuckerberg, or Bezos. But what he
   lacks in fame he makes up for in wealth. He came from a middle
   class, Long Island family. He founded Bridgewater Associates -- the
   largest hedge fund in the world -- in his two-bedroom apartment. Forbes
   suggests his net worth is \$18.1 billion. He epitomises the American
   dream.

   His book, Principles, is a manual for all his employees. In it, he
   explores what has made him successful, and provides principles that
   anyone can use to achieve their ambitions. The
   purpose of the book is to provide radical transparency.

   Transparency and failure are the two important tenets of Principles.
   Everything is recorded at Bridgewater, even executives meeting to
   discuss whether an employee deserves a raise. This opens everyone up
   to vulnerabilities. If a trader thinks his Portfolio Manager's idea is
   stupid, he should tell him bluntly. This goes for Mr. Dalio himself: he
   does not want his employees to follow his ideas just because he is the
   boss. Dalio hires smart people so they can catch mistakes, not so that
   they can follow his every move -- he often admits he is a "dumbshit."
   He requires that they challenge him.

   Transparency is necessary for growth. People need to know what others
   dislike about them so they can address those problems. If their ideas
   are mistaken, they need to be corrected straight away -- not allowed to
   bolster false confidence. This office environment may sound dystopian,
   everyone at each other's throats and throwing each other under the bus
   at the smallest misstep. Indeed, many suffer under the intense scrutiny
   and leave. Even when it is understood logically that this no-holds-barred programme works, we still feel the urge to reject
   it. Dalio insists this is all done in kindness.

   A similarly important component of success, according to Dalio, is
   failure. As he will readily admit, he fears boredom and mediocrity much
   more than he fears failure. To him, failure is not the end of the road:
   "The key is to fail, learn, and improve quickly. If you're constantly
   learning and improving, your evolutionary process will look like the
   one that's ascending." In the face of our generation's ethos -- an
   ethos of necessary perfection -- Dalio hopes to remind us that if we do
   not fail, we will never learn, and never improve.

   Failing is not enough, however. According to Dalio, we must learn to
   fail openly and honestly -- and not attempt to hide our shortcomings
   from others. "Having nothing to hide," he says, "relieves stress and
   builds trust." Ultimately, we should accept that we need not be
   perfect: "It's more important to do big things well than to do the
   small things perfectly."

   A larger picture now comes into view: the rumoured wolf-pack culture at
   Bridgewater is not one designed to forcibly weed out the failures, but
   rather, to encourage people to engage with their colleagues openly and
   honestly -- and to keep growing. Perhaps, then, this book circuitously
   aims to dispel the rumours of a cut-throat company and an unforgiving
   leader (`history is written by the victors'). But the purpose of the
   book is not exceedingly relevant: everyone who reads it will gain
   something -- perspective, nuance even -- that will help them succeed.

   The take-aways: it is important to learn to embrace failure, and to
   talk about it. If you hide failure or do not challenge yourself, you
   will never grow. Most of all: do not be afraid to be ambitious.

\textit{\href{https://www.goodreads.com/book/show/34536488-principles}{Principles}} by Ray Dalio.

  
